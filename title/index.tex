
\tableofcontents
\addcontentsline{toc}{chapter}{Inhaltsverzeichnis}

%\listoffigures
%\addcontentsline{toc}{chapter}{Abbildungsverzeichnis}

%\listoftables
%\addcontentsline{toc}{chapter}{Tabellenverzeichnis}

\chapter*{Abkürzungsverzeichnis}
\addcontentsline{toc}{chapter}{Abkürzungsverzeichnis}

%\thispagestyle{empty}
\markboth{Abkürzungsverzeichnis}{}
%
\begin{bfscript}{common}
	% enter your common and not so comman abbrev. here
	\item[AC] Alternating Current
	\item[API] Application Programming Interface
	\item[ARP] Address Resolution Protocol
	\item[ARPANET] Advanced Research Projects Agency Network
	\item[ASCII] American Standard Code for Information Interchange
	\item[AVR] 8-Bit-Mikrocontroller des Herstellers Atmel

	\item[bzw.] beziehungsweise
	\item[BSD] Berkeley Software Distribution

	\item[CPU] Central Processing Unit
	\item[CRC] Cyclic Redundancy Check
	\item[CTC] Clear Timer on Compare Match

	\item[DC] Direct Current als Gleichstrom
	\item[DNS] Domain Name System
	\item[DoD] Department of Defense

	\item[EEPROM] Electrically Erasable Programmable Read-Only Memory

	\item[FAQ] Frequently Asked Questions
	\item[FCS] Frame Check Sequence
	\item[FIFO] First In – First Out
	\item[FreeBSD] Ein freies und modernes UNIX-basiertes Betriebssystem von Berkeley Software Distribution (BSD)
	\item[FTP] File Transfer Protocol

	\item[GIF] Graphics Interchange Format
	\item[GNU/Linux] UNIX-ähnliche Mehrbenutzer-Betriebssysteme, die auf dem Linux-Kernel und wesentlich auf GNU-Software basieren

	\item[HTTP] Hypertext Transfer Protocol
	\item[Hz] Hertz (Einheit)

	\item[IANA] Internet Assigned Numbers Authority
	\item[I$^2$C] Inter-Integrated Circuit
	\item[ICF] Input Capture Flag
	\item[ICMP] Internet Control Message Protocol
	\item[ICP] Input Capture Pin
	\item[IEEE] Institute of Electrical and Electronics Engineers
	\item[IGMP] Internet Group Management Protocol
	\item[IHL] Internet Header Length
	\item[inkl.] inklusive
	\item[INT] Interrupt
	\item[IP] Internet Protocol
	\item[ISO] International Organization for Standardization
	
	\item[JTAG] Joint Test Action Group

	\item[KB] Kilobyte (Einheit)

	\item[LAN] Local Area Network
	\item[Led] Light-emitting Diode
	\item[LLC] Logical Link Control

	\item[MAC] Media Access Control
	\item[MHz] Megahertz (Einheit)
	\item[MISO] Master in, Slave out
	\item[MOSI] Master out, Slave in
	\item[MPEG] Moving Picture Experts Group

	\item[NNTP] Usenet News Transfer Protocol

	\item[OSI-Modell] Open Systems Interconnection Model

	\item[PHY] Physical Layer
	\item[POSIX] Portable Operating System Interface

	\item[QoS] Quality of Service

	\item[RAM] Random-access Memory
	\item[RARP] Reverse Address Resolution Protocol
	\item[RFC] Request for Comments
	\item[RISC] Reduced Instruction Set Computing
	\item[RJ] Registered Jack
	\item[RTC] Real Time Clock (Echtzeituhr)
	\item[RX] Pin for Receive Data

	\item[sek.] Sekunde (Einheit)
	\item[SCK] SPI Bus Serial Clock
	\item[SFD] Start Frame Delimiter
	\item[SMB] Server Message Block
	\item[SMTP] Simple Mail Transfer Protocol
	\item[SNMP] Simple Network Management Protocol
	\item[sog.] so genannt
	\item[SPI] Serial Peripheral Interface
	\item[SRAM] Static Random-access Memory
	\item[$\overline{SS}$] Slave Select 

	\item[TAP] Test Access Port
	\item[TCNT] Timer Counter
	\item[TCP] Transmission Control Protocol
	\item[Telnet] Telecommunication Network
	\item[TIFF] Tagged Image File Format
	\item[TX] Pin for Transmit Data

	\item[UART] Universal Asynchronous Receiver Transmitter
	\item[UDP] User Datagram Protocol
	\item[UNIX] Ein Mehrbenutzer-Betriebssystem und wurde im JAhr 1969 von Bell Labs bei AT\&T entwickelt
	\item[USART] Universal Synchronous/Asynchronous Receiver Transmitter
	\item[USB] Universal Serial Bus
	
	\item[V] Volt (Einheit)
 
	\item[Xerox PARC] Xerox Palo Alto Research Center

\end{bfscript}


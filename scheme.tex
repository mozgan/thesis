
%%%%%%%%%%%%%%%%%%%%%%%%%%%%%%%%%%%%%%%%%%%%%%%%%%
% 1.PACKAGES
%%%%%%%%%%%%%%%%%%%%%%%%%%%%%%%%%%%%%%%%%%%%%%%%%%

% defines thesis as report (oneside, A4 with 11pt fontsize)
\documentclass[11pt,twoside,a4paper]{ICTthesis}

\usepackage{geometry}

% formats the text accourding the set language
%\usepackage[english]{babel}
% Sprache: Deutsch
\usepackage[ngerman]{babel}   % deutsche Trennungsregeln und Übersetzung der festcodierten Überschriften


% generates indices with the "\index" command
\usepackage{makeidx}

% enables import of graphics. We use pdflatex here so do the pdf optimisation.
%\usepackage[dvips]{graphicx}
\usepackage[pdftex]{graphicx}
%\usepackage{graphicx} % Bilder
\usepackage{color} % Farben
\DeclareGraphicsExtensions{.pdf,.png,.jpg} % bevorzuge pdf-Dateien
\usepackage{pdfpages}
%\usepackage[enable-survey]{pdfpages}

% includes floating objects like tables and figures.
\usepackage{float}

% for generating subfigures with ohne indented captions
\usepackage[hang]{subfigure}
\newcommand{\subfigureautorefname}{\figurename} % um \autoref auch für subfigures benutzen
%\usepackage[all]{hypcap} % Beim Klicken auf Links zum Bild und nicht zu Caption gehen

\usepackage{nameref}

% redefines and smartens captions of figures and tables (indentation, smaller and boldface)
%\usepackage[hang,small,bf]{caption}

% enables tabstops and the numeration of lines
\usepackage{moreverb}

% enables user defined header and footer lines (former "fancyheadings")
\usepackage{fancyhdr}

% Some smart mathematical stuff
%\usepackage{amsmath}
\usepackage{mathtools}
\usepackage{amsmath,marvosym} % Mathesachen
%\usepackage{amsfonts}
%\usepackage{amssymb}
\usepackage{mathpazo} % Palatino für Mathemodus
\linespread{1.05}         % Palatino needs more leading (space between lines)
%\usepackage{mathpazo,tgpagella} % auch sehr schöne Schriften
\usepackage{setspace} % Zeilenabstand
\onehalfspacing % 1,5 Zeilen


% Package for rotating several objects
\usepackage{rotating}
\usepackage[square,sort,comma,numbers]{natbib}
%\usepackage[numbers,round]{natbib}
%\bibliographystyle{alphadin}
%\usepackage[sorting=nty,style=alphabetic]{biblatex}
\usepackage{bibgerm} % Umlaute in BibTeX
\usepackage{epsf}
\usepackage{dsfont}
%\usepackage[usenames]{color}
%\usepackage[algochapter, boxruled, vlined]{algorithm2e}

%Activating and setting of character protruding - if you like
%\usepackage[activate,DVIoutput]{pdfcprot}
% If you really need special chars...
%\usepackage[latin1]{inputenc}
\usepackage{ucs}        % Dokument in utf8-Codierung schreiben und speichern
\usepackage[T1]{fontenc} % Ligaturen, richtige Umlaute im PDF
\usepackage[utf8]{inputenc}% UTF8-Kodierung für Umlaute usw

% footnote
\usepackage{footnote}
\usepackage{tablefootnote}

\makesavenoteenv{minpage}   % If you want to include minipages. 
\makesavenoteenv{itemize}

% Hyperlinks
\usepackage[ngerman,colorlinks,hyperindex,plainpages=false,%
pdftitle={Frequenzmessung mit ATMEL-Mikrocontroller},%
pdfauthor={Mehmet Ozgan},%
pdfsubject={Bachelorarbeit},%
pdfkeywords={Frequenzmessung, Atmel, AVR, ATMega16, Optokoppler},%
pdfpagelabels,%
pagebackref,%
bookmarksopen=false%
]{hyperref}

% For the two different reference lists ...
%\usepackage{multibib}
\usepackage[resetlabels]{multibib}
\usepackage{multirow} % Tabellen-Zellen über mehrere Zeilen
\usepackage{multicol} % mehre Spalten auf eine Seite
\usepackage{tabularx} % Für Tabellen mit vorgegeben Größen
\usepackage{longtable} % Tabellen über mehrere Seiten
\usepackage{array}

% TikZ und plot
\usepackage{ifthen}
\usepackage{tikz}
\usepackage{pgf}
\usepackage{pgffor}
%\usepgfmodule{shapes}
\usepgfmodule{plot}
\usetikzlibrary{decorations}
\usetikzlibrary{arrows}
%\usetikzlibrary{snakes}
\usetikzlibrary{decorations.pathmorphing}


%%%%%%%%%%%%%%%%%%%%%%%%%%%%%%%%%%%%%%%%%%%%%%%%%%
% 2.Settings
%%%%%%%%%%%%%%%%%%%%%%%%%%%%%%%%%%%%%%%%%%%%%%%%%%

% redifine the paragraph command.
\makeatletter
\renewcommand\paragraph{\@startsection{paragraph}{4}{\z@}%
                                    {3.25ex \@plus1ex \@minus.2ex}%
                                    {0.3em} %-1em}%
                                    {\normalfont\normalsize\bfseries}}

\renewcommand\subparagraph{\@startsection{subparagraph}{5}{\parindent}%
                                       {3.25ex \@plus1ex \@minus .2ex}%
                                       {-1em}%
                                      {\normalfont\normalsize\bfseries}}
\makeatother

%Enables numbers at subsubsections without inserting them into the toc.
\setcounter{secnumdepth}{3}

% generates the index (command for the subprocessor)
\makeindex

% default path to your pictures
\graphicspath{{pictures/}}

% Counter for the maximum number of "Floatobjects" at the beginning of the page.
\setcounter{topnumber}{2}
% Redefines the maximum area which floats my consume at the beginning of the page.
\def\topfraction{.8}
% Counter for the maximum numbers of floats at the end of the page
\setcounter{bottomnumber}{2}
% Redefines the maximum area which floats my consume at the end of the page.
\def\bottomfraction{.5}
% Maximal number of floats per page
\setcounter{totalnumber}{8}
% minimal amount of text per page
\def\textfraction{.2}
% Redefinition: minimal amount of floats in percent per floatpage.
\def\floatpagefraction{.6}
% no indentation at paragraphs
\setlength{\parindent}{0pt}

% part of the caption package: extra 20pts left and right of captions.
%\setlength{\captionmargin}{20pt}

% sets the page layout
\setlength{\oddsidemargin}{4mm}
\setlength{\evensidemargin}{-6mm}
\setlength{\textwidth}{162mm} 
\setlength{\textheight}{230mm}
\setlength{\topmargin}{-5mm}
%\addtolength{\headsep}{12pt}

%part of the "float" Packages:
\floatstyle{plain}
% define a new floating object
\floatname{example}{Example}

%\newfloat{example}{hbtp}{loe}[chapter]
%\floatplacement{figure}{hbt}
%\floatplacement{table}{htb}

% enables a "\dollar" command (returns $)
\newcommand{\dollar}{\char36}

% Script for abbreviations
% defines a new environment with one arguement
\newenvironment{bfscript}[1] {
 % defines as list
 \begin{list}
 % No labelmarks!
 {}
 {%\settowidth{\labelwidth}{\bf #1}
 \setlength{\labelwidth}{7em}
  % sets the left margin to 0 because there is no labelmark
  \setlength{\leftmargin}{\labelwidth}
  % add labelsep (0, no labelmark) to the margin
  \addtolength{\leftmargin}{\labelsep}
  % Separation of paragraphs in one topic
  \parsep 0.0ex plus 0.2ex minus 0.2ex
  % Separation of two topics
  \itemsep -0.5ex
  % sets the label to: boldface and fills with whitspace to the text
  \renewcommand{\makelabel}[1]{\parbox[t]{6em}{##1\hfill}}}}
 {\end{list}
}

% PDF-Settings
\def\pdfBorderAttrs{/Border [0 0 0] } % No border arround Links
%\pdfcompresslevel=9
\hypersetup{colorlinks,linkcolor=blue,filecolor=red,urlcolor=black,citecolor=blue}
% PDF-Kompression
\pdfminorversion=5
\pdfobjcompresslevel=1
\usepackage[final]{microtype} % mikrotypographische Optimierungen
\usepackage{url}
\usepackage{pdflscape} % einzelne Seiten drehen können
%\usepackage{lmodern}   % verwenden der "Latin Modern" ("Computer Modern"++)



%%%%%%%%%%%%%%%%%%%%%%%%%%%%%%%%%%%%%%%%%%%%%%%%%%
% 3.HYPENATION
%%%%%%%%%%%%%%%%%%%%%%%%%%%%%%%%%%%%%%%%%%%%%%%%%%

% enter special rules here!
\hyphenation{gleich-zeitig para-meter}

% Quellcode
\usepackage{minted}
\usepackage{listings} % für Formatierung in Quelltexten
\usepackage{courier}
\definecolor{grau}{gray}{0.25}
%\lstset{
% language=C,
% extendedchars=true,
% basicstyle=\tiny\ttfamily,
% %basicstyle=\footnotesize\ttfamily,
% tabsize=8,
% keywordstyle=\textbf,
% commentstyle=\color{grau},
% stringstyle=\textit,
% numbers=left,
% numberstyle=\tiny,
% % für schönen Zeilenumbruch
% breakautoindent  = true,
% breakindent      = 2em,
% breaklines       = true,
% postbreak        = ,
% prebreak         = \raisebox{-.8ex}[0ex][0ex]{\Righttorque},
%}
\lstset{
    language=C,
    basicstyle=\tiny\ttfamily, % \ttfamily\small,
    breaklines=true,
    prebreak= \raisebox{-.8ex}[0ex][0ex]{\Righttorque}, %\raisebox{0ex}[0ex][0ex]{\ensuremath{\hookleftarrow}},
    frame=topline,
    showtabs=false,
    showspaces=false,
    showstringspaces=false,
    keywordstyle=\color{red}\bfseries,
    stringstyle=\color{green!50!black},
    commentstyle=\color{gray}\itshape,
    numbers=left,
    captionpos=t,
    escapeinside={\%*}{*)}
}



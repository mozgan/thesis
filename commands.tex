

\newcommand{\mat}[1]{\ensuremath{\mathbf #1}}
\newcommand{\set}[1]{\ensuremath{\mathbf #1}}
\newcommand{\cset}[1]{\ensuremath{\mathbf{\mathcal #1}}}
\renewcommand{\vec}[1]{\ensuremath{\mathbf #1}}

\newcommand{\nth}[1]{\ensuremath{#1^\mathrm{th}}}
\newcommand{\fst}[1]{\ensuremath{#1^\mathrm{st}}}
\newcommand{\snd}[1]{\ensuremath{#1^\mathrm{nd}}}

\newcommand{\argmax}[1]{\ensuremath{\arg\hspace{-0.4ex}\max_{\hspace*{-3.0ex}#1}}}
\newcommand{\argmin}[1]{\ensuremath{\arg\min_{\hspace*{-4.0ex}#1}}}

\newcommand{\argmaxi}[1]{\ensuremath{\arg\hspace{-0.4ex}\max_{#1}}}
\newcommand{\argmini}[1]{\ensuremath{\arg\hspace{-0.4ex}\min_{#1}}}

% \newcommand{\argmin}[1]{\ensuremath{\begin{array}[t]{c} \arg \min \\
% \vspace*{-0.1ex} #1 \end{array}}}

\newcommand{\NP}{\ensuremath{\mathcal{NP}}}
\newcommand{\PP}{\ensuremath{\mathcal{P}}}
\newcommand{\e}[2]{\ensuremath{\{#1,#2\}}}
\newcommand{\tup}[1]{\ensuremath{\langle#1\rangle}}
\newcommand{\bigO}[1]{\ensuremath{\mathcal{O}\left(#1\right)}}

\newcommand{\trans}[1]{\ensuremath{{#1}^\top}}
\newcommand{\diag}[1]{\ensuremath{\mathrm{diag}\left(#1\right)}}

\newcommand{\eq}[1]{equation \ref{#1}}
\newcommand{\Eq}[1]{equation \ref{#1}}
\newcommand{\fig}[1]{figure \ref{#1}}
\newcommand{\Fig}[1]{figure \ref{#1}}
\newcommand{\chap}[1]{chapter \ref{#1}}
\newcommand{\Chap}[1]{chapter \ref{#1}}
\newcommand{\sect}[1]{section \ref{#1}}
\newcommand{\Sect}[1]{section \ref{#1}}

\newcommand{\bydefn}{\ensuremath{\stackrel{\bigtriangleup}{=}}}
\newcommand{\elmat}[2]{\ensuremath{#1 \odot #2}}

\newcommand{\prune}[1]{\ensuremath{\mathrm{prune}\left(#1\right)}}

\newcommand{\labelfig}[2]{\parbox[b]{0.2in}{\Large#1\normalsize\vspace{#2}}}
% \newcommand{\labelfig}[1]{\parbox[b]{0.2in}{#1\vspace{1.8in}}}

% \newcommand{\emptyset}{\ensuremath{\O}}

\renewcommand{\Re}{\mathbb{R}}

\newcommand{\incfig}[3]{\ifx\pdfoutput\undefined
                          \epsfig{#1.eps,#2,#3}
                        \else
                          \epsfig{#1.eps,#2,#3}
                        \fi}
          
% Eigene Befehle %%%%%%%%%%%%%%%%%%%%%%%%%%%%%%%%%%%%%%%%%%%%%%%%%
\newcommand{\todo}[1]{
      {\colorbox{red}{ TODO: #1 }}
}

\newcommand{\todotext}[1]{
      {\color{red} TODO: #1} \normalfont
}

\newcommand{\info}[1]{
      {\colorbox{blue}{ (INFO: #1)}}
}

\newcommand{\infotext}[1]{
      {\color{blue}{ (INFO: #1)}}
}

% Hinweis auf Programme in Datei
\newcommand{\datei}[1]{
      {\ttfamily{#1}}
}

\newcommand{\code}[1]{
      {\small \ttfamily {#1}}
}

